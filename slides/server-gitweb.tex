\subsection{Gitweb}

\begin{frame}
\frametitle{Gitweb}
\begin{itemize}
	\item Nakon postave HTTP servera, možemo postaviti visualizaciju repozitorija, za pregledan pristup preko web preglednika
	\item Git dolazi sa skriptom GitWeb koja se često koristi u tu svrhu
	\item Možemo provjeriti kako nam izgleda repozitorij uz naredbu:

	\$ git instaweb --httpd=(webserver)

	\item Naredba pokreće privremeni HTTPD server s pregledom našeg repozitorija, a gasi se:

	\$ git instaweb --http=(webserver) --stop
\end{itemize}
\end{frame}

\begin{frame}
\frametitle{Gitweb - postava}
\begin{itemize}
	\item Ako želimo postaviti Gitweb, trebamo ga preuzeti:
	\begin{itemize}
		\item Iz Gitovog izvornog koda, ili
		\item Neke distribucije Linuxa imaju \textit{gitweb} paket koji se može preuzeti
	\end{itemize}
	\item Dodajemo Gitweb skriptu na server uz pomoć Apache-a te možemo preko web preglednika vidjeti naš repozitorij
	\item Više o postavi na \href{https://git-scm.com/book/en/v2/Git-on-the-Server-GitWeb}{poveznici}
\end{itemize}
\end{frame}