
\section{Git na serveru}

\subsection {Postavljanje Git-a na server}

\begin{frame}

\frametitle{Postavljanje Git-a na server}

\begin{itemize}

	\item Koraci:

	\begin{itemize}

		\item izvoženje postojećeg repozitorija u novi prazan repozitorij u kojemu nema radnog direktorija

		\item Primjer: $ git clone --bare moj_projekt moj_projekt.git

		\item stavljanje praznog repozitorija u server

		\item Primjer: $ scp -r moj_projekt.git korisnik@git.primjer.com:/srv/git

		\item Svi korisnici koji imaju pristup serveru mogu klonirati naš repozitorij

	\end{itemize}

\end{itemize}

\end{frame}



\begin{frame}

\frametitle{SSH Public Key}

\begin{itemize}

	\item SSH Public Key - način autentifikacije mnogih Git servera

	\item Svaki korisnik koji radi na nekom projektu na serveru mora imati svoj 'javni ključ'

	\item Korisnici mogu naći svoj 'javni ključ' u svojem ~/.ssh direktoriju

	\item Ako nemaju 'javni ključ' moraju ga generirati u Git-u izvođenjem "ssh.keygen" programa

\end{itemize}

\end{frame}