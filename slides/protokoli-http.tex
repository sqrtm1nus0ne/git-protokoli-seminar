\subsection{HTTP protokoli}

\begin{frame}
\frametitle{HTTP protokoli}
\begin{itemize}
	\item Jednostavni za korisnika
	\item Funkcionira slično SSH i Git protokolima
	\item Dvije vrste:
	\begin{itemize}
		\item Smart HTTP (noviji)
		\item Dumb HTTP (stariji)
	\end{itemize}
	\item Naredba za kloniranje:

	\$ git clone https://primjer.com/projekt.git
\end{itemize}
\end{frame}


\subsection{Smart HTTP}
\begin{frame}
\frametitle{Smart HTTP}
\begin{itemize}
	\item Koristi HTTPS portove i razne HTTP mehanizme za autentifikaciju
	\item Lakše za koristiti od SSH protokola, koristi korisničko ime i lozinku za prijavu umjesto SSH ključeva
	\item Zbog Smart HTTP protokola, URL koji koristimo za gledati repozitorij preko web preglednika je isti kojim možemo klonirati repozitorij
\end{itemize}
\end{frame}


\subsection{Dumb HTTP}
\begin{frame}
\frametitle{Dumb HTTP}
\begin{itemize}
	\item Stariji od Smart HTTP-a, danas se koristi kao pričuvna metoda spajanja na repozitorij, ako Smart HTTP ne uspije
	\item Izuzetno jednostavan za koristiti
	\item Svatko tko ima pristup web serveru na kojem se nalazi repozitorij, može taj isti repozitorij klonirati
	\item Ovaj protokol je najbolje koristiti samo za read-only verziju repozitorija
\end{itemize}
\end{frame}