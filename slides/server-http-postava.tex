\subsection{Smart HTTP}

\begin{frame}
\frametitle{Smart HTTP}
\begin{itemize}
	\item SSH protokol nam omogućuje autentificirani pristup, dok nam Git protokol daje slobodan pristup
	\item HTTP protokolom možemo oboje istovremeno
	\item Pomoću Smart HTTP-a možemo postaviti server tako da provjerava može li klijent komunicirati preko HTTP-a, odnosno ako ne može, da komunicira s njim na starije metode
	\item Također će nam ovo omogućiti pristup repozitoriju preko web preglednika
\end{itemize}
\end{frame}


\subsection{Postava Smart HTTP-a}

\begin{frame}
\frametitle{Postava Smart HTTP-a}
\begin{itemize}
	\item Za postavu HTTP protokola, koristimo Apache za CGI server (više o postavi na \href{https://git-scm.com/book/en/v2/Git-on-the-Server-Smart-HTTP}{poveznici})
	\item Uz Apache možemo izabrati hoće li provjeravati jesu li korisnici autentificirani te može li im posluživati repozitorij
	\item Autentificirani korisnici imaju read/write pristup, dok neautentificirani imaju read-only pristupS
	\item Autentificiranim korisnicima će lozinke biti zapisane u \textit{.htpasswd} datoteci za provjeru
	\item Također je bitno postaviti SSL enkripciju radi sigurnosti
\end{itemize}
\end{frame}

