\section{Protokoli}
\subsection{O protokolima za Git}

\begin{frame}
\frametitle{Protokoli za Git}

Četiri vrste:
\begin{itemize}
	\item Lokalni protokol
	\item HTTP protokoli:
	\begin{itemize}
		\item Smart HTTP
		\item Dumb HTTP
	\end{itemize}
	\item SSH protokol
	\item Git protokol
\end{itemize}
\end{frame}


\subsection{Lokalni protokol}

\begin{frame}
\frametitle{Lokalni protokol}
\begin{itemize}
	\item Najjednostavniji protokol korišten na istom uređaju
	\item Koristan ako svi članovi imaju pristup zajedničkom disku ili (manje vjerojatno) koriste isto računalo
	\item Kloniranje repozitorija na lokalnom uređaju se jednostavno napravi naredbom:

	\$ git clone /put/do/repozitorija/projekt.git

	ili

	\$ git clone file:///put/do/repozitorija/projekt.git
\end{itemize}
\end{frame}

\begin{frame}
\frametitle{Lokalni protokol}
\begin{itemize}
	\item Također se lokalni repozitorij može dodati postojećem projektu:

	\$ git remote add lokalni\_projekt /put/do/repozitorija/projekt.git

	\item Ovako se mogu primati i slati datoteke na repozitorij putem imena "lokalni\_projekt" kao da ga koristimo preko mreže
\end{itemize}
\end{frame}